\documentclass[usenames,dvipsnames]{article}

%\renewcommand{\familydefault}{\sfdefault}

\usepackage{standalone}
\usepackage{pdfpages}
\usepackage{upgreek}
\usepackage[english]{babel}
\usepackage{geometry}

\usepackage{color}
\usepackage{xcolor}
\usepackage{float}
\usepackage{tabularx}
\usepackage{booktabs}
\usepackage{pgfplots}
\usepackage{amsmath}
\usepackage{listings}
\usepackage{amssymb}
\usepackage{amsfonts}
\usepackage{siunitx}
\usepackage{tikz}
\usepackage{graphics} % for pdf, bitmapped graphics files
\usepackage{graphicx}
\usepackage{exsheets}
\usepackage{exsheets-listings}
\usepackage{cprotect}
\usepackage{algorithm}
\usepackage{algorithmicx}
\usepackage[noend]{algpseudocode}
\usepackage[pdfusetitle]{hyperref}
\usepackage[shortlabels]{enumitem}
\usepackage{filecontents}
\usepackage{multirow}
\usepackage{multicol}
\usepackage{setspace}
\usepackage{xparse}
\usepackage{fancyhdr}
\usepackage{datetime}
\usepackage{totcount}
\usepackage{fancyvrb}
\usepackage{minted}
% Needed to manage fonts
\ifxetex
  \usepackage{fontspec}
  % To support LaTeX quoting style
  \defaultfontfeatures{Ligatures=TeX}
\else
  \usepackage[LGR, T1]{fontenc}
  % Replace by the encoding you are using
  \usepackage[utf8x]{inputenx}
\fi
% Needed to manage math fonts
%\usepackage{unicode-math}
%\renewcommand{\familydefault}{\sfdefault}
% Needed to use icons from font-awesome
% (https://github.com/posquit0/latex-fontawesome)
\usepackage{fontawesome}

%\usepackage{../partials/fontawesome5}

%\usepackage{fontawesome}
%\usepackage{fontawesome5}
%\usepackage{academicons}


\usetikzlibrary{backgrounds,patterns,arrows,arrows.meta,calc,intersections,shapes,positioning,decorations.pathreplacing,decorations.markings,decorations.pathmorphing}
\usepackage{multicol}
\usepackage{subcaption}
\sisetup{output-decimal-marker={,},exponent-product=\cdot}

    
    
    
\DeclareSIUnit\atm{atm}
\DeclareSIUnit\dioptre{D}

\def\BState{\State\hskip-\ALG@thistlm}

% Setup the page geometry
\geometry{a4paper,top=15mm, bottom=15mm, footskip=5mm}


\definecolor{TitleColor}{rgb}{0.65,0.04,0.07}
\definecolor{NumberColor}{rgb}{0.02,0.04,0.48}

\DeclareInstance{exsheets-heading}{fancy}{default}{
toc-reversed = true ,
indent-first = true ,
vscale = 2 ,
pre-code = \IfInsideQuestionT{\rule{\linewidth}{1pt}} ,
post-code =\IfInsideQuestionT{\rule{\linewidth}{1pt}} ,
title-format = \large\scshape\color{Maroon},
subtitle-format = \large\scshape\color{rgb:red,0.65;green,0.04;blue,0.07} ,
number-format = \large\bfseries\color{rgb:red,0.02;green,0.04;blue,0.48} ,
points-format = \itshape ,
join = { number[r,B]title[l,B](.333em,0pt);
title[r,B]subtitle[l,B](1em,0pt)
} ,
attach =
{
main[hc,vc]number[hc,vc](0pt,0pt) ;
main[l,vc]subtitle[hc,vc](\marginparsep,0pt)
}
}



\DeclareInstance{exsheets-heading}{block-subtitle}{default}{
vscale = 2 ,
pre-code = \rule{\linewidth}{1pt} ,
post-code = \rule{\linewidth}{1pt} ,
title-format = \large\scshape\color{Maroon} ,
number-format = \large\bfseries\color{Blue} ,
subtitle-format = \large\scshape\color{black} ,
join = {
title[r,B]number[l,B](.333em,0pt) ;
title[r,B]subtitle[l,B](1em,0pt)
} ,
attach = {
main[l,vc]title[l,vc](0pt,0pt) ;
main[r,vc]points[l,vc](\marginparsep,0pt)
},
}

\DeclareQuestionClass{textbook}{textbooks}

\SetupExSheets{
  headings = fancy,
  question/print = true ,
  solution/print = true }
 % counter-format = se.qu ,
%  counter-within = section ,
  %question/pre-hook = \rule{\textwidth}{1pt},


\hypersetup{
	colorlinks = true, 
	breaklinks = true, 
	bookmarks = true,
	bookmarksnumbered = true,
	urlcolor = blue, 
	linkcolor = blue, 
	citecolor=blue,
	linktoc=page, 
	pdftitle={}, 
	pdfauthor={\textcopyright Author}, 
	pdfsubject={}, 
	pdfkeywords={}, 
	pdfcreator={pdfLaTeX}, % PDF Creator
	pdfproducer={IEEE} }



\lstset{
frame=single,
xleftmargin=20pt,
numbers=left,
numberstyle=\small,
tabsize=2,
breaklines,
showspaces=false,
showstringspaces=false,
language=r,
keywordstyle=\color{blue}\ttfamily,
stringstyle=\color{red}\ttfamily,
commentstyle=\color{green}\ttfamily,
morecomment=[l][\color{magenta}]{\#},
basicstyle=\small\ttfamily
}


\RenewQuSolPair{question}[name=Problem]{solution}


\newfontfamily\crimsonfont[
  Path=partials/fonts/,
  UprightFont=*-Roman,
  ItalicFont=*-Italic,
  BoldFont=*-Bold,
  BoldItalicFont=*-BoldItalic,
]{Crimson}

%\setmainfont[
%  Path=../fonts/,
%  UprightFont=*-Roman,
%  ItalicFont=*-Italic,
%  BoldFont=*-Bold,
%  BoldItalicFont=*-BoldItalic,
%]{Crimson}

\setmainfont[
  Path=partials/fonts/,
  UprightFont=*-Regular,
  ItalicFont=*-It,
  BoldFont=*-Bold,
  BoldItalicFont=*-BoldIt
]{MinionPro}


\setsansfont[
  Path=partials/fonts/,
  UprightFont=*-Regular,
  ItalicFont=*-Italic,
  BoldFont=*-Bold,
  BoldItalicFont=*-BoldItalic,
]{Roboto}
 


\fancyhf{}
\fancyfoot[L]{\footnotesize{(Last build on \today\ at \currenttime)}}
\pagestyle{fancy}

\usemintedstyle{monokai}
\definecolor{bg}{rgb}{0.95,0.95,0.95}
\setmintedinline{bgcolor=Black}
\newcommand{\CppVerb}[1]{\mintinline{c++}{#1}}
\tikzset{point/.style={circle,fill,black!80,inner sep=0pt,minimum size=#1,opacity=0.9}}
\tikzset{point/.default=3pt}\tikzset{vector/.style={line width=1pt,postaction={decorate,decoration={markings,mark=at position 1 with {\arrow{latex}}}}}}
\tikzset{block/.style={rectangle,fill=black!30,draw,minimum size=#1,opacity=0.9,align=center}}
\tikzset{block/.default=15pt}\tikzset{ball/.style={circle,fill=black!30,draw,minimum size=#1,opacity=0.9}}
\tikzset{ball/.default=5pt}\tikzset{pulley/.style={draw=black,line width=0.2pt,circle,minimum size=#1,inner sep=0pt,fill=black!10}}
\tikzset{pulley/.default=20pt}\tikzset{rod/.style={line width=2pt}}
\tikzset{rope/.style={line width=1pt}}
\tikzset{spring/.style={decorate,decoration={coil,amplitude=5pt,segment length=#1,aspect=0.3}}}
\tikzset{spring/.default=5pt}\tikzset{wall/.style={black!10,pattern=north east lines,opacity=0.3}}
\tikzset{ray/.style={line width=0.8pt,postaction={decorate,decoration={markings,mark=at position 0.5 with {\arrow{>}}}}}}
\tikzset{arrow/.style={-latex}}
\tikzset{object/.style={line width=1pt,orange,-latex}}
\tikzset{image/.style={line width=1pt,blue,-latex}}
\tikzset{doublearrow/.style={<->,>=latex,thick}}
\tikzset{brace/.style={decorate,decoration={brace,amplitude=#1}}}
\tikzset{brace/.default=5pt}


\renewcommand{\familydefault}{\rmdefault}

%YOUR NAME HERE
\newcommand{\studentname}{Asem Mohamed Alaaeldin}
%YOUR SECTION HERE INSTEAD OF 0
\newcommand{\SN}{ 0 }
%YOUR BENCH NUMBER HERE INSTEAD OF 0
\newcommand{\BN}{ 0 }


\author{\studentname ~ Sec. \SN ~ BN. \BN}
\date{Sunday 12\textsuperscript{th} April, 2020}
\title{Data Structures and Algorithms [SBE201] (Spring 2020)\\ Report 1\\~\\
{\small  Linked Lists }}

\makeatletter
\@addtoreset{question}{section}
  
\hypersetup{
    pdfauthor={{Asem Alaa}},
    pdftitle={{\@title}},
    pdfsubject={Data Structures and Algorithms [SBE201] (Spring 2020)},
    pdfkeywords={Report 1, Linked Lists}
    }
    
\makeatother


\begin{document}

\maketitle

\newminted{cpp}{bgcolor=Black,frame=lines,framesep=2mm,baselinestretch=1.2,linenos}

\section{Problem Set}

\subsection{Linked List Size}




\cprotEnv \begin{question}
\begin{cppcode}
struct IntegerNode
{
  int data;
  IntegerNode *next;
};

int size( IntegerNode *front )
{

}
\end{cppcode}

\begin{enumerate}[A)]
\item Implement the function \CppVerb{size} that returns the size of a given linked list (count of elements).
\item Provide a time complexity estimate using the Big-O notation.
\item Can you provide a recursive version of the \CppVerb{size} function?
\end{enumerate}
\end{question}

\cprotEnv \begin{solution}

\begin{enumerate}[A)]
\item  \mbox{}\\

\begin{cppcode}
int size( IntegerNode *front )
{
  int count = 0;
  auto tempt = front;
  while( temp != nullptr ) 
  {
    ++count;
    temp = temp->next;
   }
  else count;
}
\end{cppcode}
\item $O(n)$
\item \mbox{}\\
\begin{cppcode}
int size( IntegerNode *front )
{
  if( front == nullptr ) return 0;
  else return 1 + size( front->next );
}
\end{cppcode}
\end{enumerate}


\end{solution}


\subsection{Linked List Operations}




\cprotEnv \begin{question}
\begin{cppcode}
#include <iostream>
struct IntegerNode
{
  int data;
  IntegerNode *next;
};
void funx(node* front)
{
  if(front == nullptr) return;
  fun1(front->next);
  std::cout << front->data << " ";
}
\end{cppcode}

\begin{enumerate}[A)]
\item What does the function \CppVerb{funx} do?
\item What is the output would be if the input linked list is represented in order as: \CppVerb{5->90->300->7->55}
\item What is the time complexity of such a function.
\end{enumerate}
\end{question}

\begin{solution}

\begin{enumerate}[A)]
\item prints the elements of the LL in \textbf{reversed order}.
\item \CppVerb{"55 7 300 90 5"}
\item $O(n)$
\end{enumerate}

\end{solution}



\subsection{Doubly-Linked List}



\cprotEnv \begin{question}
\begin{cppcode}
struct IntegerNode
{
  int data;
  IntegerNode *next;
  IntegerNode *back;
};

struct IntegersLL
{
  IntegerNode *front;
};

void insertAt( IntegersLL &list , int index, int data )
{
  
}
\end{cppcode}
\begin{enumerate}[A)]
\item Implement a function \CppVerb{insertAt} to insert an element at arbitrary \CppVerb{index} in a \textbf{linked list}.
\item Provide a visual illustratoin to the steps in order to support that operation.
\end{enumerate}
\end{question}
\cprotEnv \begin{solution}

\begin{enumerate}[A)]
\item  \mbox{}\\
\begin{cppcode}
void insertAt( IntegersLL &list , int index, int data )
{
  auto temp = list.front;
  for( int i = 0; i < index; ++i ) temp = temp->next;
  auto node = new IntegerNode{ data , temp , temp->prev };
  if( temp->next ) temp->next->prev = node;
  if( temp->prev ) temp->prev->next = node;
}
\end{cppcode}

\item 

\includegraphics{imgs/insertAt.png}


\end{enumerate}

\end{solution}


\subsection{Circular Linked List}





\cprotEnv \begin{question}
\begin{cppcode}
struct IntegerNode
{
  int data;
  IntegerNode *next;
};

struct IntegersLL
{
  IntegerNode *front;
};

void pushFront( IntegerLL &list, int data )
{
    list.front = new node{ data , list.front };
}

node *backNode( IntegerLL &list )
{
    node *temp = list.front;
    while( temp->next != nullptr )
        temp = temp->next;
    return temp;
}

void *pushBack( IntegerLL &list, double data )
{
    if( list.front == nullptr )
        return pushFront( list , data );
    else
    {
        node *back = backNode( list );
        back->next = new node{ data , nullptr };
    }
}

void removeBack( IntegerLL &list )
{ 
    if( isEmpty( list ))
        return;
    else if( list.front->next == nullptr )
        removeFront( list );
    else
    {
        IntegerNode *prev = list.front;
        while( prev->next->next != nullptr )
            prev = prev->next;
        delete prev->next;
        prev->next = nullptr;
    }
}

void printLL( IntegerLL &list )
{
    node *current = list.front;
    while( current != nullptr )
    {
        std::cout << current->data;
        current = current->next;
    }
}
\end{cppcode}

The functions: \CppVerb{pushFront}, \CppVerb{backNode}, \CppVerb{pushBack}, \CppVerb{removeBack}, and \CppVerb{printLL} are a implemented earlier for a regular linked list.
How would you change each function to work properly for a circular linked list that uses only a \textbf{front} pointer.
\end{question}

\begin{solution}

\begin{itemize}
\item Check \href{https://github.com/sbme-tutorials/sbme-tutorials.github.io/blob/master/2020/data-structures/snippets/report1/cdll/main.cpp}{an implementation of circular doubly-linked list (template class)}
\item Check \href{https://github.com/sbme-tutorials/sbme-tutorials.github.io/blob/master/2020/data-structures/snippets/report1/cll/main.cpp}{an implementation of circular singly-linked list (template class)}

\end{itemize}
\end{solution}

\end{document}
